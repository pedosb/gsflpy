\documentclass[a4paper]{article}

\usepackage[utf8]{inputenc}

%To use url
\usepackage{hyperref}
%\url{http://www.wikibooks.org}
%\href{http://www.wikibooks.org}{wikibooks home}


\author{Nelson Neto \and Pedro Batista \and Aldebaro Klautau}

\title{Experiencia de Rescore de Lattices Usando o gsflpy\_15best}


\begin{document}

\date{\today}

\maketitle

\begin{abstract}
Neste documento apresentaremos a experiencia em rescore utilizando
o software gsflpy em conjunto com o classificador Multilayer Perceptron
do \href{http://www.cs.waikato.ac.nz/ml/weka/}{Weka}. O rescore foi
realizado apenas em frames que continham pelo menos dois modelos
específicos ($v[4]$ e $ow1[4]$). O conjunto de teste apresentava inicialmente
um erro por sentença de XX\% e após o rescore o erro caio para XX\%.
A leitura desse documento assume conhecimento prévio de proposta.pdf e 
gsflpy\_manul.
\end{abstract}

\section{Introdução}

Casa amarela

\bibliographystyle{play}
\bibliography{bib.bib}

\end{document}
