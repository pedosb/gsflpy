\documentclass[a4paper]{article}

\usepackage[utf8]{inputenc}

%To use url
\usepackage{hyperref}
%\url{http://www.wikibooks.org}
%\href{http://www.wikibooks.org}{wikibooks home}


\author{Pedro Batista \and Nelson Neto \and Aldebaro Klautau}

\title{gsflpy Manual}


\begin{document}

\date{\today}

\maketitle

\begin{abstract}
Este documento tem como objetivo, capacitar o leitor para usar o pacote
gsflpy, este é capaz de percorrer lattices, fazer rescore baseado em 
detecção de confusões entre frames, dentre outras funcionalidades.
Antes de seguir neste documento recomenda-se a leitura do proposta.
\end{abstract}

\section{Instalação}
\subsection{Pré-requisitos}
Para a instalação assume-se que os seguintes pacotes de terceiros estão
instalados no sistema:
\begin{itemize}
   \item \href{http://python.org}{Python 2.*}
   \item \href{http://www.jython.org/}{Jython 2.*}
   \item \href{http://matplotlib.sourceforge.net/}{matplotlib}
   \item \href{http://www.cs.waikato.ac.nz/ml/weka/}{Weka 3.6}
\end{itemize}

\subsubsection{Instalando Python}
O interpretador Python é disponível em pacote apt para Ubuntu, para
instala-lo o comando abaixo é suficiente.
\begin{verbatim}
$sudo apt-get install python
\end{verbatim}

\subsubsection{Instalando Jython}
O interpretador Jython é disponível em pacote apt para Ubuntu, para
instala-lo o comando abaixo é suficiente.
\begin{verbatim}
$sudo apt-get install jython
\end{verbatim}

\subsubsection{Instalando matplotlib}
A matplotlib é uma biblioteca para plotar gráficos em Python, no Ubuntu
o pacote do apt pode ser instalado como mostrado abaixo.
\begin{verbatim}
$sudo apt-get install python-matplotlib
\end{verbatim}

\subsubsection{Instalando Weka}
Para usar o GsflReScore precisamos do pacote Weka no \texttt{CLASSPATH}. Para
isso podemos fazer o download da versão mais recente em:
\url{http://www.cs.waikato.ac.nz/ml/weka/}. Uma vez feito o download
podemos descompactar o arquivo como se seque, assumindo que o download
foi feito na pasta \texttt{\~{}/Downloads}.
\begin{verbatim}
$cd ~
$mkdir programs
$cd programs
$unzip -x Downloads/weka-3-6-3.zip
\end{verbatim}
Para colocar o Weka no seu \texttt{CLASSPATH} e possibilitar seu uso pelo gsflpy,
deve-se adicionar a seguinte linha no seu arquivo \texttt{\~{}/.bashrc} (se não existir
você pode cria-lo). Lembre-se de trocar \texttt{user} pelo seu nome de usuário.
\begin{verbatim}
export CLASSPATH=$CLASSPAH:/home/user/programs/weka-3-6-3/weka.jar
\end{verbatim}
Pronto agora basta reiniciar seu \texttt{bash}, ou se preferir digitar o comando abaixo.
\begin{verbatim}
$source ~/.bashrc
\end{verbatim}

\bibliographystyle{play}
\bibliography{bib.bib}

\end{document}
